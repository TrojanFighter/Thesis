\label{conclusions}

%\chapter{CONCLUSION AND FUTURE WORK}
\section{TA problem}

In this part, we search for a path that the Target can move on to escape from the Attacker. All the evasion techniques depend on the time of the turn that the Target makes when it detects the Attacker (Missile) and the objective is to maximize the Missile acceleration till the Missile power bleed. We choose the escaping trajectory as a polynomial with unknown coefficients, then decide the values of these coefficients so as to make the Missile exert a maximum acceleration to bleed its power as fast as possible before it reaches the Target. Optimization is achieved via the techniques of Monte-Carlo simulation and genetic algorithms. We establish a Graphic User Interface (GUI) “Guidance toolbox” containing our guidance law and several types of maneuvers. This toolbox is an open-source program for the development and addition of other guidance laws and maneuvers. So anybody can add some guidance laws or maneuvers as a future work.
Then A novel paradigm of games played by humans versus computers is introduced and exemplified by a game handling the sophisticated and complex task of solving our classical pursuit-evasion problem that has many automated algorithmic solutions based on mathematically-elaborate game-theoretic techniques. The proposed strategy requests a small number of ordinary people to play a computer game, which is constructed to result in a correct solution and (as incentive to players) is enjoyable. The preliminary results obtained via our first game version Evasion-1 are reasonable and encouraging. In fact, human players unknowingly obtained a target acceleration of a Barrel-Roll shape, one of the best known policies for practical periodic maneuvers. The results suggest the production of enhanced versions of the game with more desirable outcomes. Specifically, in our next version (Evasion-2), the plane will no longer be destined to be intercepted by the missile, since we will impose the restriction that the missile has a limited reservoir of fuel that could be drained as a result of clever target maneuver. The ultimate goal is to test human brain capability to work collectively to produce a new guidance law which could possibly compete in accuracy with the existing ones, while surpassing them in the simplicity of the way it is produced. 

\section{TAD problem}
This part offered a unified analytic solution of the $TAD$ problem in which an Attacker is pursuing a Target while attempting to evade a Defender. The thesis reviews and extends the work that has recently appeared in \cite{pachter2014active,garcia2015active,garcia2015escape}, beside making the following new contributions:

\begin{enumerate}
\item The thesis covers all cases for the ratio $\gamma$ of the Attacker's speed w.r.t the Defender's speed. It treats the case of a slow Defender $\gamma>1$ for the first time, and presents this case along with the case of a fast Defender ($\gamma<1$) discussed in \cite{garcia2015active} and the case of a similar Defender ($\gamma=1$), discussed earlier in \cite{pachter2014active,garcia2015escape}.
\item The thesis demonstrates a striking increase in complexity when $\gamma\neq1$ compared with the case $\gamma=1$. It also demonstrates some sort of \textit{duality} between the two cases of ($\gamma<1$) and ($\gamma>1$).
\item The thesis develops novel analytic expressions for the Voronoi diagrams for bordering the escape regions when ($\gamma<1$) or ($\gamma>1$). These expressions are more complex than the ones obtained in [3] for ($\gamma=1$), and reduce to it as a limiting case.
\item The thesis offers a tutorial exposition of the $TAD$ problem, uses simple arguments of plane geometry to develop the necessary Apollonius circle, utilizes equalities rather than inequalities in developing Voronoi diagram, and pays careful attention to the inadvertent inclusion of extraneous solutions so as to justify their subsequent rejection.
\item The thesis supplements its analysis with extensive computations for the critical speed ratio, Voronoi diagrams, and the optimal interception points. The results obtained encompass all possible values of $\gamma$, and they reduce to the already available results for $\gamma=1$. Results for the trajectories and optimal interception points obtained agree with those obtained by the numerical solution of a two-point boundary value problem (TPBVP) utilizing Pontryagin's Maximum Principle \cite{garcia2015active}.     
\end{enumerate}  

Some possible extensions of the current work that warrant further exploration include:
\begin{enumerate}
\item Further analysis of the quartic equation obtained for the Voronoi diagram when $\gamma\neq1$, with an aim to \textit{split} it into two factors representing the rejected and accepted branches of the diagram.
\item Investigation of the sixth-degree complex polynomial equation for the optimal interception angle to get some \textit{insight} about its six roots, and to find a better way for \textit{selecting} desirable root.
\item Relaxation of some of the assumption used in this study. In particular, it is very interesting to consider the possibility of \textit{variable} rather than constant speeds the three agents.
\item Addition of an element of \textit{uncertainty} to the computation. For example, we might assume that the initial positions of the three agents are not deterministic but \textit{stochastic} or \textit{fuzzy}.
\item Extension of the current work to a more general situation involving several Targets, several Attackers and/or several Defenders.     
\end{enumerate} 