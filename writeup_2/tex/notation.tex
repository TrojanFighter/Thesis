\section{Assumptions}
\begin{enumerate}
\item The speeds $V_{T},V_{A},$ and $V_{D}$ of the Target, Attacker, and Defender are constant.
\item The Attacker missile is faster than the Target aircraft, i.e., $\alpha\equiv \dfrac{V_{T}}{V_{A}}<1$ (In the case $\alpha\geq1$, the Target is guaranteed to survive by following an optimal strategy, even without assistance from the Defender).
\item There are three distinct cases for the ratio of the speed of the Attacker to that of the Defender $\gamma=\dfrac{1}{\beta}=\dfrac{V_{A}}{V_{D}}$ 
\begin{itemize}
\item $\gamma<1$ (fast Defender) discussed earlier in Garcia et el. \cite{garcia2015active}, and extended, expounded, simplified and exposed herein.
\item $\gamma =1$ (same speed Defender) discussed in Garcia et el. \cite{pachter2014active,garcia2015escape}, and extended, expounded, simplified and exposed herein.
\item $\gamma>1$ (slow Defender), which is a novel case, discussed herein for the first time, and unified with the two previous cases.  
\end{itemize}

\item The Defender intercepts the Attacker if their separation becomes zero (point capture).
\item The optimal trajectories of the three agents are straight lines.
\item A Cartesian frame is attached to the initial positions $A$ and $D$ of the Attacker and Defender in such a way the $X$-axis is the infinite extension of the straight segment $\overline{AD}$ and the $Y$-axis is the perpendicular bisector of $\overline{AD}$.
\end{enumerate}

\section{Notation}
\begin{itemize}
\item $A$: Initial position of the Attacker.
\item $D$: Initial position of the Defender.
\item $T$: Initial position of the Target.
\item $T'$ Terminal position of the Target, i.e., its position at the time the Defender intercepts the Attacker.
\item $\alpha=\dfrac{V_{T}}{V_{A}}$ (assumed $<1$, otherwise the target trivially survives).
\item $\gamma=\dfrac{1}{\beta}=\dfrac{V_{A}}{V_{D}}$ (studied for a fast Defender ($\gamma<1$), a similar Defender ($\gamma=1$), and a slow Defender ($\gamma>1$)).
\item $u, v, w$: Aim points on the $AD$ Apollonius circle by the Attacker, Target and Defender, respectively. 
\end{itemize}

\section{Nomenclature}
\subsection{The $AD$ Apollonius circle}
The locus of a point such that the ratio of its distance from the initial positions $\boldsymbol{A}=(x_{A},0)$ and $\boldsymbol{D}=(-x_{A},0)$ of the Attacker and Defender, respectively, is a fixed ratio $\gamma=\dfrac{V_{A}}{V_{D}}$. This circle degenerates into the perpendicular bisector of $\overline{AD}$ if $\gamma=1$.
Points on the circumference of this circle are reached simultaneously by the Attacker and Defender. For $\gamma<1$, $\boldsymbol{A}$ belongs to the interior of this circle and points within the circle are reached by the Attacker before the Defender. For $\gamma>1$, $\boldsymbol{D}$ belongs to the interior of this circle and points within the circle are reached by the Defender before the Attacker. For $\gamma=1$, the $AD$ circle becomes of infinite radius and degenerates into a straight line, namely the perpendicular bisector of $\overline{AD}$. In this case, $\boldsymbol{A}$ belongs to the R.H.S. of the $XY-$plane where points are reached by the Attacker before the Defender, while $\boldsymbol{D}$ belongs to the L.H.S. of the $XY-$plane where points are reached by the Defender before the Attacker.   

\subsection{The $TA$ Apollonius circle} 
The locus of a point such that the ratio of its distances from the initial positions $\boldsymbol{T}=(x_{T},y_{T})$ and $\boldsymbol{A}=(x_{A},0)$, of the Target and Attacker, respectively, is a fixed ratio $\alpha= \dfrac{V_{T}}{V_{A}}$ strictly less than 1. Naming of the $AD$ and $TA$ Apollonius circles herein follows a common practice in mathematical circles \cite{ayoub2003proving,ayoub2006circle,partensky2008circle} and is opposite to the style used by Garcia et al. \cite{pachter2014active,garcia2015active,garcia2015escape}.

\subsection{Region reachable by the Defender before the Attacker (Reachability region $R_r$)}
For $\gamma<1$, $R_r$ is the exterior of the $AD$ Apollonius circle.\\
For $\gamma=1$, $R_r$ is the L.H.S. of the $XY-$plane.\\
For $\gamma>1$, $R_r$ is the interior of the $AD$ Apollonius circle. 

Figure \ref{Rr2} illustrates this region as a shaded region to which point $D$ belongs.

\begin{figure}
\centering
\subfigure[$\gamma<1$.]{\includegraphics[width=0.5\textwidth]{fig/drawing4_1a.pdf}}\label{2_g<1}
\quad
\subfigure[$\gamma=1$.]{\includegraphics[width=0.5\textwidth]{fig/drawing4_1b.pdf}}\label{2_g=1}
\\
\subfigure[$\gamma>1$.]{\includegraphics[width=0.75\textwidth]{fig/drawing4_1c.pdf}}\label{2_g>1}
\caption{The reachability region $R_r$ (one including $\boldsymbol{D}$ whose points are reached by the Defender before the Attacker) is shown shaded.}
\label{Rr2}
\end{figure}

\subsection{Region of guaranteed Target's escape (Escape region $R_e$)}
The escape region $R_e$ is the set of all coordinate pairs $(x,y)$ such that if the Target initial position $\boldsymbol{T}=(x_{T},y_{T})$ is inside this region, then it is guaranteed to escape the Attacker if both the Target and Defender implement their corresponding optimal strategies. This set is a strict superset of the set of points reachable by the Defender before the Attacker ($R_e\supseteq R_r$). The boundary of this set is called a Voronoi Diagram.

\subsection{The critical speed ratio $\overline{\alpha}$}
A lower limit on the speed ratio $\gamma=\dfrac{V_{T}}{V_{A}}$, attained when the $TA$ Apollonius circle is tangent to the $AD$ Apollonius circle (or to the perpendicular bisector of $\overline{AD}$ in the degenerate case $\gamma=1$).   

\subsection{Escape Condition}
For a given speed ratio $\alpha=\dfrac{V_{T}}{V_{A}}$ and a given Attacker's initial position $\boldsymbol{A}=(x_{A},0)$, the escape condition is that the $TA$ Apollonius circle (of center $O_2$ and radius $r_2$) intersects the $AD$ Apollonius circle (of center $O_1$ and radius $r_1$). This happens for $\gamma<1$ when
\begin{equation}
\lvert \boldsymbol{O_{1}}-\boldsymbol{O_{2}}\rvert+r_{2}>r_{1}
\end{equation}
It happens for $\gamma>1$ when
\begin{equation}
\lvert \boldsymbol{O_{1}}-\boldsymbol{O_{2}}\rvert-r_{2}<r_{1}
\end{equation}
In the degenerate case of $\gamma=1$, it is required that the $TA$ Apollonius circle intersects the $Y-$axis, i.e., 
\begin{equation}
r_{2}>the\ abscissa\ of\ \boldsymbol{O_{2}}
\end{equation}
