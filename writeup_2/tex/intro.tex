% 
%
% INTRODUCTION -- High Level, both problems
%
%

This thesis deals with two scenarios of active target defense involving:
\begin{enumerate}
	\item Two-agent pursuit-evasion: a Target (aircraft) in opposition to an Attacker (missile). The Target tries to evade (avoid being captured) the Attacker. This problem will be referred to herein as the TA problem as it concerns the Target ($T$) and Attacker ($A$).
	\item Three-agent pursuit-evasion where each agent has a specific role.
	A two-agent team consists of a Target (aircraft) and a Defender (missile) cooperating in opposition to an Attacker (missile).
	The Target tries to evade the Attacker and avoid being captured by him.
	The Defender cooperates with and assists the Target by trying to intercept (capture and destroy) the Attacker before the latter captures the Target.
	This problem will be referred to herein as the TAD problem as it concerns a triad consisting of the Target ($T$), Attacker ($A$), and Defender ($D$).
\end{enumerate}


The following two sections pertain to a brief introduction and literature survey for each of the $ TA $ and $ TAD $ problems.

\section{The $ TA $ problem}

%In the TA problem, we search for a path that the Target can move on it to escape from the Attacker. All the evasion techniques depend on the time of the turn that the Target makes when it detects the Attacker (Missile) and the objective is to maximize the Missile acceleration till the Missile power bleed. We choose the escaping trajectory as a polynomial with unknown coefficients, then decide the values of these coefficients so as to make the Missile exert a maximum acceleration to bleed its power as fast as possible before it reaches the Target. We explain the meaning of proportional navigation, and subsequently simulate two- dimensional proportional-navigation equations using MATLAB and Simulink.


Solutions to the TA problem are called homing missile guidance laws, and the most popular among them is a control law called proportional navigation \cite{adler1956missile,becker1990closed,ghose1994generalization}, which is analogous to classical proportional control. The underlying basic concept is that if the direct Target-Attacker line-of-sight time rate is diminished, i.e., if the direct TA Line-of-Sight ceases to change its direction then (for a non-maneuvering, constant velocity Target) the Attacker is on a collision course. If the Target is considered smart or maneuvering, then variations to the proportional navigation are needed so as to minimize the miss distance. These variations have been given optimal-control interpretations through linear quadratic Gaussian (LQG) formulations, and a variety of other techniques \cite{gutman2012optimal,shinar1980three,nesline2012new,anderson2012comparison,pastrick2012guidance,rodin1987pursuit,rodin1989pursuit,lin1991modern,cochran1991analytical,le1998determining,creaser1998generation,siouris2004missile,lechevin2004lyapunov,lee2007guidance,breivik2008guidance,rusnak2008guidance, shinar2009meeting,lin2010development,shneydor1998missile,pham2012fuzzy,white2012advances,li2014fuzzy,cho2014optimal,grinfeld2015minimal,zarchan1999ballistic,zarchan2000tracking,zarchan2011kill,zarchan2002tactical}.

The literature has investigated manifold variations of the TA problem (including the TAD problem, producing a large amount of results, and a huge number of publications (see, \textit{e.g.}, \cite{rodin1987pursuit,rodin1989pursuit,chung2011search,weiss2017combined,garcia2017optimal}. The study of pursuit evasion games can be traced back to the von Neumann’s hide-and-seek game \cite{flood1972hide}, where a hider chooses one cell of a two-dimensional grid in which to hide himself and a seeker chooses a subset of cells of the grid (usually one row and one column) in which to seek the hider. If the seeker selects the cell chosen by the hider, then the hider is captured and the seeker wins the game. Otherwise, the hider wins the game. Starting from this seminal work, adversarial pursuit evasion settings have been modeled by a variety of mathematical techniques. Notable among these are the optimal-control formulation and the differential-game one \cite{anderson2012comparison,turetsky2003missile,gutman20103d}. 

The fundamental conceptual difference between missile guidance laws based on optimal control theory and those based on differential game theory is in the assumptions made by the guidance laws on the future trajectory and maneuvering capabilities of the Target. Optimal control theory assumes that the future maneuver strategy of the Target is completely defined, either in open-loop or closed-loop form \cite{anderson2012comparison}. The feedback nature of missile guidance laws allows the missile to make corrections for inaccurate predictions of the maneuvers of the Target. The optimal-control formulation is appropriate only when future maneuver time history, or strategy, of the Target is known or can be justifiably assumed or accurately predicted. One possible perspective is to design strategies that maximize performance of the Attacker against a worst-case Target \cite{chung2011search}. In such a setting, the Target is usually considered to be of a finite speed, complete awareness of the location and intent of the Attacker, and full knowledge of the conflict environment. Such a method guarantees the success of the pursuit, defined, for example, by capture of the Target in a finite time. However, this powerful-adversary model may yield solutions that are too conservative in practical applications. A better alternative is to use probabilistic formulations addressing average-case behaviours \cite{chung2011search}. 

The most suitable (and presumably most successful) mathematical framework for analyzing conflicts controlled by two independent agents is in the realm of dynamic or differential games. Thus, the scenario of intercepting a maneuverable Target has to be formulated as a zero-sum pursuit–evasion game \cite{turetsky2003missile}. The roles of the game players are clearly defined, the interceptor (Attacker) is the pursuer and the Target is the evader. The natural cost function of such a zero-sum game is the miss distance (the distance of the closest approach, or in other words the smallest norm of the separation vector), to be minimized by the pursuer and maximized by the evader. The game solution provides simultaneously the missile’s guidance law (the optimal pursuer strategy), the “best” Target maneuver (the optimal evader strategy), and the resulting guaranteed miss distance (the value of the game). As a consequence, the game solution provides a guidance law that is robust with respect to the Target maneuver structure.

Many prominent extensions of the TA problem are currently problems of hot research. In addition to the celebrated TAD problem (to be fully introduced in section \ref{TADsection}), we give a glimpse of problems of contemporary interest in the following (far-from-conclusive) list

\begin{itemize}
	\item Three-dimensional pursuit-evasion \cite{adler1956missile,shinar1980three,lin2010development,gutman20103d},
	\item Fuzzy guidance laws \cite{pham2012fuzzy,li2014fuzzy},
	\item Non-conventional or modern approaches for interception \cite{nesline2012new,shinar2009meeting},
	\item Lyapunov-based non-linear guidance \cite{lechevin2004lyapunov}.
	\item Incorporation of probabilistic uncertainty in the location, behavior, and/or sensor observations of the Target.
	
\end{itemize}



\section{The $ TAD $ problem}  \label{TADsection}

The $TAD$ problem constitutes a dynamic differential game \cite{ho1965differential,isaacs1954differential,meier1969new,hsueh2007differential,yi2010improved,bressan2010noncooperative,perelman2011cooperative,battistini2014differential,yavin2014pursuit} and is of interest in aerospace, control, and robotics engineering. We will consider herein recent formulations and treatments of this problem \cite{pachter2014active,garcia2015active,garcia2015escape,garcia2014cooperative,garcia2015cooperative,garcia2015cooperative2}, though there exist other formulations and treatments of it that span almost half a century \cite{boyell1976defending,shneydor1977comments,rusnak2005lady,de2010analysis,rusnak2011guidance,fuch2011encouraging,scott2013pursuit,rubinsky2013three,oyler2014pursuit}.
Table \ref{tableTAD} shows several variants of this generic problem in a variety of settings, contexts or disciplines. The $TAD$ problem is a generalization of the classical problem of a single pursuer and a single evader \cite{anderson1978model,miller1994co,cliff1995co,pekalski2004short,zarchan2002tactical}. In fact, the $TAD$ is essentially a duplication of this classical problem, since in the $TAD$ problem, the Attacker plays the double role of being a pursuer for the Target, and at the same time an evader for the Defender \cite{rusnak2008guidance}.
The $TAD$ problem is also a special case of a more general pursuit-evasion problem in which there are multiple Attackers and multiple Defenders  \cite{hagedorn1976differential,kim2001multiagent,fuchs2010cooperative,pan2012pursuit,ragesh2014analysis}. Many common threads are shared by all these problems, such as the solution of differential games \cite{ho1965differential,isaacs1954differential,meier1969new,hsueh2007differential,yi2010improved,bressan2010noncooperative,perelman2011cooperative,battistini2014differential,yavin2014pursuit} and the construction of Apollonius circles \cite{ayoub2003proving,ayoub2006circle,partensky2008circle,fulton2015conflict} and Voroni diagrams \cite{gowda1983dynamic,aurenhammer1991voronoi,cheung2007pursuit,gavrilova2008generalized,majdandzic2008computation,bakolas2010optimal,bakolas2010zermelo,bakolas2011optimal}.


\begin{table}
	\caption{A variety of settings or contexts for the same generic problem}
\begin{tabular}{ |c||c|c|c|c| } 
\hline
Area & Agent 1 & Agent 2 & Agent 3 & References \\
 \hline
 \hline
 Aerospace & Target & Attacker (missile) & Defender (missiles) & \cite{pachter2014active,garcia2015active,garcia2015escape,garcia2014cooperative,garcia2015cooperative}\\
 \hline 
 Biology & Prey & Predator & Protector & \cite{de2010analysis,oyler2014pursuit}\\
 \hline 
 Society & Lady & Bandits & Bodyguards & \cite{rusnak2005lady}\\ 
 \hline
 Criminology & Robber & Policemen/Cops & Gangsters & \cite{cheung2007pursuit}\\ 
  \hline
\end{tabular}

\label{tableTAD}
\end{table}  

\section{Thesis Contribution}
As stated earlier, this thesis has contributions to both the $ TA $ and the $ TAD $ problems.

In the TA problem, we search for a path that the Target can move on to escape from the Attacker. All the evasion techniques depend on the time of the turn that the Target makes when it detects the Attacker (Missile) and the objective is to maximize the Missile acceleration till the Missile power bleed. We choose the escaping trajectory as a polynomial with unknown coefficients, then decide the values of these coefficients so as to make the Missile exert a maximum acceleration to bleed its power as fast as possible before it reaches the Target. We explain the meaning of proportional navigation, and subsequently simulate two- dimensional proportional-navigation equations using MATLAB and Simulink. Optimization is achieved via the techniques of Monte-Carlo simulation and genetic algorithms. We establish a Graphic User Interface (GUI) “Guidance toolbox” containing our guidance law and several types of maneuvers. This toolbox is an open-source program for the development and addition of other guidance laws and maneuvers. We also construct a mathematically-correct game of target-attacker and let many people play it taking the target side. We find the best escape maneuver by collecting and analyzing data of the human escape maneuver. The game is developed using Unity, a free readily-available cross-platform game engine.

The thesis also offers a unified analytic treatment of the $TAD$ problem based on the construction of two Apollonius circles.
The treatment includes all possibilities of the ratio between the speeds of the Attacker and Defender. Note that the case of a slow Defender appears here for the first time, while the treatment of the cases of fast Defender or similar Defender is extended and augmented with novel results and new insights. A critical condition is derived from which the two following important entities are obtained
\begin{itemize}
\item the critical Target speed normalized w.r.t. the Attacker's speed so as to be a dimensionless quantity,
\item the Voronoi diagram bordering the safe or escape region for the Target.
\end{itemize}

Optimal strategies are also studied, and are shown to obey a complex sixth-degree polynomial when the Defender differs in speed from the Attacker. This polynomial reduces to a real fourth-degree polynomial when the Defender and Attacker are similar in speed.
Beside unifying previously published results in a common setting, the thesis also simplifies all computations by using intuitionistic plane-geometric arguments rather than the more tedious analytic-geometric manipulations. 
Moreover, the thesis extends existing results by adding some novel results, thereby giving a complete picture of all cases of interest.
The analysis in this thesis is supplemented by extensive computations using MATLAB to solve the complex high-order polynomial equations and to plot the Voronoi diagrams under a variety of pertinent parameters. The numerical results and plots obtained allow useful and insightful interpretations and are in exact agreement with numerical solution of the corresponding two-point boundary value problem.

\bigskip 

\section{Thesis Outline}
The thesis consists of 9 chapters. The current chapter is introduction and literature survey. Chapter \ref{introTA} is an introduction of the $ TA $ problem and  devoted to a brief discussion of the guidance law used here. Chapter \ref{optimization_EM} constitutes a prominent cornerstone of the thesis as it reports extensive simulation results for the $ TA $ problem obtained via Matlab and simulink. Chapter \ref{game} discusses a new game-based methodology for discovering optimal escape maneuver. It is followed by second problem of the thesis on the $ TAD $ problem, which consists of chapters \ref{ATD_problem}-\ref{simulation}. Chapter \ref{ATD_problem} introduces the $ TAD $ problem, lists the assumptions, notation and nomenclature used therein, discusses the general concepts and properties of an Apollonius circle \cite{ayoub2003proving,ayoub2006circle,partensky2008circle,fulton2015conflict}, and then specializes these to the cases of the $AD$ Apollonius circle and the $TA$ Apollonius circle. Chapter \ref{speed_ratio} derives a criticality condition which constitutes a quadratic equation that is later solved for the critical dimensionless ratio between the Target speed and Attacker speed. The same criticality condition is then rephrased in Chapter \ref{voronoi} as a relation between the initial coordinates $x_T$ and $y_T$ of the Target. This relation leads to the Voronoi diagram bordering the \textit{safe} or \textit{escape} region $R_e$ of the Target. This Voronoi diagram is published for the first time when the Defender is fast or slow. It includes as a special case the much simpler diagram for a similar Defender that has already appeared in \cite{garcia2015escape}. Chapter \ref{simulation} explores the simulation of the optimal heading angle. The thesis is concluded with chapter \ref{conclusions} that summarizes the thesis contributions and points out possible directions for further work. 


%Chapter \ref{strategies} develops optimal strategies for the three constituents of the differential game.
