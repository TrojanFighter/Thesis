This thesis deals with two scenarios of target defense. We first present several methodologies to find the optimal escape maneuver for a target against an attacking missile. We simulate two-dimensional proportional-navigation using MATLAB and Simulink. Optimization is achieved via the techniques of Monte-Carlo simulation and genetic algorithms. We establish a Graphic User Interface (GUI) “Guidance toolbox” containing our guidance law and several types of maneuvers. This toolbox is an open-source program for the development and addition of other guidance laws and maneuvers. We also construct a mathematically-correct game of target-attacker and let many people play it taking the target side. We find the best escape maneuver by collecting and analyzing data of the human escape maneuver. The game is developed using Unity, a free readily-available cross-platform game engine. We then consider the case when the target is being helped by a defender. We offer a unified analytic treatment of this active defense problem via the construction of two Apollonius circles, considering all possibilities of the ratio between the speeds of the attacker and defender. A criticality condition is derived, from which we obtain the critical target speed and the Voronoi diagram bordering the safe or escape region for the target optimal strategies. Our numerical results and plots allow useful and insightful qualitative interpretations. Next, we find the optimal heading angles to be followed by the target so as to stay in the safe region. We use Hamiltonian equations to formulate an exact two-point boundary value problem that is solved numerically, yielding results verifying our earlier results.  
