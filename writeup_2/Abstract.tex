
This thesis deals with two scenarios of active target defense. The first scenario is a two-agent pursuit-evasion problem that involves a Target (aircraft) in opposition to an Attacker (missile). The Target tries to evade the Attacker and avoid being captured by him. This first problem will be referred to herein as the TA problem as it concerns the Target (T) and Attacker(A). The second scenario is a three-agent pursuit-evasion problem that involves three agents, the Target, the Attacker and the Defender. The Attacker missile pursues a Target aircraft that is being helped by a Defender missile which tries to intercept the Attacker before it reaches the Target. This second problem will be referred to herein as the TAD problem as it concerns the Target (T), the Attacker(A), and the Defender (D). We first consider the TA problem, in which we search for a path that the Target can move on it to escape from the Attacker. All the evasion techniques depend on the time of the turn that the Target makes when it detects the Attacker (Missile) and the objective is to maximize the Missile acceleration till the Missile power bleed. We choose the escaping trajectory as a polynomial with unknown coefficients, then decide the values of these coefficients so as to make the Missile exert a maximum acceleration to bleed its power as fast as possible before it reaches the Target. We explain the meaning of proportional navigation, and subsequently simulate two- dimensional proportional-navigation equations using MATLAB and Simulink.
 After dealing with the TA problem, the thesis shifts to the TAD problem. In this latter problem, a differential game arises in which a team is formed by the Target and the Defender which cooperate to maximize the separation between the Target and the point where the Defender intercepts the Attacker, while the Attacker tries to minimize this separation. This thesis offers a unified analytic treatment of the aforementioned problem based on the construction of two Apollonius circles. The treatment includes all possibilities of the ratio between the speeds of the Attacker and Defender. A criticality condition is derived from which two important entities are obtained, namely: (a) the critical Target speed normalized w.r.t. the Attacker speed, and (b) the Voronoi diagram bordering the safe or escape region for the Target Optimal strategies are also studied, and are shown to obey a complex sixth-degree polynomial when the Defender differs in speed from the Attacker. This polynomial reduces to a real fourth-degree polynomial when the Defender and Attacker are similar. Beside unifying previously published results in a common setting, this thesis simplifies all computations by using intuitionistic plane-geometric arguments rather than the more tedious analytic-geometric manipulations. Moreover, the thesis extends existing results by adding some novel results, thereby giving a complete picture of all cases of interest. The analysis in this thesis is supplemented by extensive computations using MATLAB to solve the complex high-order polynomial equations and to plot the Voronoi diagrams under a variety of pertinent parameters. The numerical results and plots obtained allow useful and insightful interpretation and are in exact agreement with numerical solution of the corresponding two-point boundary value problem.