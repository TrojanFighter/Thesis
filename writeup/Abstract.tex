This thesis deals with an important pursuit-evasion problem that involves three agents, the Target the Attacker and the Defender. The Attacker missile pursues a Target aircraft that is being helped by a Defender missile which tries to intercept the Attacker before it reaches the Target. A differential game arises in which a team is formed by the Target and the Defender which cooperate to maximize the separation between the Target and the point where the Defender intercepts the Attacker, while the Attacker tries to minimize this separation. This thesis offers a unified analytic treatment of the aforementioned problem based on the construction of two Apollonius circles.

The treatment includes all possibilities of the ratio between the speeds of the Attacker and Defender. A criticality condition is derived from which two important entities are obtained, namely: (a) the critical Target speed normalized w.r.t. the Attacker's speed, and (b) the Voronoi diagram bordering the safe or escape region for the Target
Optimal strategies are also studied, and are shown to obey a complex sixth-degree polynomial when the Defender differs in speed from the Attacker. This polynomial reduces to a real fourth-degree polynomial when the Defender and Attacker are similar.
Beside unifying previously published results in a common setting, this thesis simplifies all computations by using intuitionistic plane-geometric arguments rather than the more tedious analytic-geometric manipulations. 
Moreover, the thesis extends existing results by adding some novel results, thereby giving a complete picture of all cases of interest.
The analysis in this thesis is supplemented by extensive computations using MATLAB to solve the complex high-order polynomial equations and to plot the Voronoi diagrams under a variety of pertinent parameters.
The numerical results and plots obtained allow useful and insightful interpretation and are in exact agreement with numerical solution of the corresponding two-point boundary value problem.