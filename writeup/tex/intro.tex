% 
%
% INTRODUCTION -- High Level, both problems
%
%

This thesis deals with two scenario of active target defense involving:
\begin{enumerate}
	\item Two-agent pursuit-evasion: a Target (aircraft) in opposition to an Attacker (missile). The Target tries to evade the Attacker and avoid being captured by him. This problem will be referred to herein as the TA problem as it concerns the Target ($T$) and Attacker($A$).
	\item Three-agent pursuit-evasion where each agent has a specific role.
	A two-agent team consists of a Target(aircraft) and a Defender(missile) cooperating in opposition to an Attacker(missile).
	The Target tries to evade the Attacker and avoid being captured by him.
	The Defender cooperates with and assists the Target by trying to intercept(capture and destroy) the Attacker before the latter captures the Target \cite{pachter2014active,garcia2015active,garcia2015escape,garcia2014cooperative,garcia2015cooperative,garcia2015cooperative2}.
	This problem will be referred to herein as the TAD problem as it concerns a triad consisting of the Target($T$), Attacker($A$), and Defender($D$).
\end{enumerate}

The $TAD$ problem constitutes a dynamic differential game \cite{ho1965differential,isaacs1954differential,meier1969new,hsueh2007differential,yi2010improved,bressan2010noncooperative,perelman2011cooperative,battistini2014differential,yavin2014pursuit} and is of interest in aerospace, control, and robotics engineering. We will consider herein recent formulation and treatments of this problem \cite{pachter2014active,garcia2015active,garcia2015escape,garcia2014cooperative,garcia2015cooperative,garcia2015cooperative2}, though there exist other formulations and treatments of it that span almost half a century \cite{boyell1976defending,shneydor1977comments,rusnak2005lady,rusnak2008guidance,de2010analysis,rusnak2011guidance,fuch2011encouraging,scott2013pursuit,rubinsky2013three,oyler2014pursuit}.
Table \ref{tableTAD} shows several variants of this generic problem in a variety of settings, contexts or disciplines. The $TAD$ problem is a generalization of the classical problem of a single pursuer and a single evader \cite{anderson1978model,miller1994co,cliff1995co,pekalski2004short,zarchan2002tactical}. In fact, the $TAD$ is essentially a duplication of this classical problem, since in the $TAD$ problem, the Attacker plays the double role of being a pursuer for the Target, and at the same time an evader for the Defender \cite{rusnak2008guidance}.
The $TAD$ problem is also a special case of a more general pursuit-evasion problem in which there are multiple Attackers and multiple Defenders  \cite{hagedorn1976differential,kim2001multiagent,fuchs2010cooperative,pan2012pursuit,ragesh2014analysis}. Many common threads are shared by all these problems, such as the solution of differential games \cite{ho1965differential,isaacs1954differential,meier1969new,hsueh2007differential,yi2010improved,bressan2010noncooperative,perelman2011cooperative,battistini2014differential,yavin2014pursuit} and the construction of Apollonius circles \cite{ayoub2003proving,ayoub2006circle,partensky2008circle,fulton2015conflict} and Voroni diagrams \cite{gowda1983dynamic,aurenhammer1991voronoi,cheung2007pursuit,gavrilova2008generalized,majdandzic2008computation,bakolas2010optimal,bakolas2010zermelo,bakolas2010optimal}.


\begin{table}
\begin{tabular}{ |c||c|c|c|c| } 
\hline
Area & Agent 1 & Agent 2 & Agent 3 & References \\
 \hline
 \hline
 Aerospace & Target & Attacker(missile) & Defender(missiles) & \cite{pachter2014active,garcia2015active,garcia2015escape,garcia2014cooperative,garcia2015cooperative}\\
 \hline 
 Biology & Prey & Predator & Protector & \cite{de2010analysis,oyler2014pursuit}\\
 \hline 
 Society & Lady & Bandits & Bodyguards & \cite{rusnak2005lady}\\ 
 \hline
 Criminology & Robber & Policemen/Cops & Gangsters & \cite{cheung2007pursuit}\\ 
  \hline
\end{tabular}
\caption{A variety of settings or contexts for the same generic problem}
\label{tableTAD}
\end{table}  

This thesis offers a unified analytic treatment of the aforementioned $TAD$ problem based on the construction of two Apollonius circles.
The treatment includes all possibilities of the ratio between the speeds of the Attacker and Defender.Note that the case of a slow Defender appears here for the first time, while the treatment of the cases of fast Defender or similar Defender is extended and augmented with novel results and new insights. A criticality condition is derived from which the two following important entities are obtained
\begin{itemize}
\item the critical Target speed normalized w.r.t. the Attacker's speed so as to be a dimensionless quantity,
\item the Voronoi diagram bordering the safe or escape region for the Target.
\end{itemize}

Optimal strategies are also studied, and are shown to obey a complex sixth-degree polynomial when the Defender differs in speed from the Attacker. This polynomial reduces to a real fourth-degree polynomial when the Defender and Attacker are similar in speed.
Beside unifying previously published results in a common setting, this thesis simplifies all computations by using intuitionistic plane-geometric arguments rather than the more tedious analytic-geometric manipulations. 
Moreover, the thesis extends existing results by adding some novel results, thereby giving a complete picture of all cases of interest.
The analysis in this thesis is supplemented by extensive computations using MATLAB to solve the complex high-order polynomial equations and to plot the Voronoi diagrams under a variety of pertinent parameters. The numerical results and plots obtained allow useful and insightful interpretations and are in exact agreement with numerical solution of the corresponding two-point boundary value problem.

\bigskip 

The organization of the remainder of this thesis is as follows. Chapter 2 lists the assumptions, notation and nomenclature used herein. Chapter 3 discusses the general concepts and properties of an Apollonius circle \cite{ayoub2003proving,ayoub2006circle,partensky2008circle,fulton2015conflict}, and then specializes these to the cases of the $AD$ Apollonius circle and the $TA$ Apollonius circle. Chapter 4 derives a criticality condition which constitutes a quadratic equation that is later solved for the critical dimensionless ratio between the Target speed and Attacker speed. The same criticality condition is then rephrased in Chapter 5 as a relation between the initial coordinates $x_T$ and $y_T$ of the Target. This relation leads to the Voronoi diagram bordering the \textit{safe} or \textit{escape} region $R_e$ of the Target. This Voronoi diagram is published for the first time when the Defender is fast or slow. It includes as a special case the much simpler diagram for a similar Defender that has already appeared in \cite{garcia2015escape}. Chapter 6 develops optimal strategies for the three constituents of the differential game. Chapter 7 concludes the thesis and points out possible directions for further work.  
