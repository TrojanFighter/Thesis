% ===================================================
\section{Problem Statement}
% ===================================================
We have a two-agent pursuit-evasion problem, a Target (aircraft) in opposition to an Attacker (missile). The Target tries to evade the Attacker and avoid being captured by him. We will try to find a simple technique for target evasion, so we should have a look at evasion techniques.

\subsection{Evasion techniques}
There is some notes we should know before talking about the techniques 
\begin{itemize}
	\item Missile is faster than the aircraft, but cannot turn tighter than the aircraft, so it takes a longer bath.
	\item In order to pull as tight turn as a fighter aircraft, missile has to pull amount of g that is amount of g’s aircraft can pull multiplied by difference in speed squared.
	\item Missile always attempts to lead the target. Thus if target changes heading, it will be hard pressed to correct.
	\item  Main problem with evading missiles is their speed, which makes timing somewhat difficult.
\end{itemize}

%\begin{table}
%	\begin{tabular}{ |c|c|c|c| } 
%		\hline  %-----------------------
%   If missile is fired head-on at BVR range
% & Jinking (useful at short ranges) 
% & Climb (useful at longer ranges)
% & Fourth tactic \\
%%		\hline  %---------------
%		\hline  %---------------
%		
% Turn hard to either left or right so as to fly at roughly 90 degrees angle to attacking aircraft (This forces missile to bleed off the energy and to lead the target).
% Once target aircraft makes a hard turn to reverse a direction, missile with its far larger turn circle – will be unable to compensate.
%	
%		 &  %||||||||||
%		
% Aircraft must be positioned so that it is at angle (30-60 degrees is optimum) relative to missile’s flight path.
% Once missile gets closer, aircraft will make a hard turn in opposite direction.
%As there is a lag between aircraft changing the direction and missile following (for several reasons, most important of which is missile’s inertia), this will cause missile to head in wrong direction until it manages to correct, and also to bleed off the energy
% Missile will fly past the aircraft and miss.
%	
%	 &   %|||||||||||||
%	 
%	 
% Since at long range missile will have burned out its engine, it will rely on inertia to keep it flying, and climbing will mean that it will bleed off energy rapidly.
% Once missile reaches a close range (maybe around 1,500 meters), dive for the ground, then pull up (This will allow pilot to gain energy and using it to evade the missile).
%	 
%	  &    %|||||||||||
%	   
% place the missile at 3 o’clock or 9 o’clock position
% maintain sufficient turn to keep the missile 
% This tactic forces the missile to execute a continuous turn, bleeding the energy entire time, making it easier to outturn the missile once it comes close.
%	   \\
	%		\hline  %------------------------------- 
	%		 \begin{figure}[htb]
	%		 	\centering
	%		 	\includegraphics[scale = 0.3]{fig/evasion tech1.jpg}
	%		 \end{figure}
	
	% & %||||||||||||||||||||
	 
	%	\begin{figure}[htb]
	%		 \centering
	%		 \includegraphics[scale = 0.3]{fig/evasion tech2.jpg}
	%	\end{figure} 
	
	%    &    %|||||||||||||||||
	
	%\begin{figure}[htb]
	%	\centering
	%	\includegraphics[scale = 0.3]{fig/evasion tech3.jpg}
	%\end{figure}
	
	
	%    &    %||||||||||||||||||| 
	 
	 
	%\begin{figure}[htb]
	%\centering
	%\includegraphics[scale = 0.3]{fig/evasion tech4.jpg}
	%\end{figure}\\
	%		\hline   %----------------
	
	
	%	\end{tabular}
	%	\caption{A variety of evasion techniques}
	%	\label{evasion tech}
	%\end{table}  



%Why use polynomials?
Now we want to search for a path that the target can move on it to escape from the attacker. All the evasion techniques depend on the time of the turn that target makes when it detect the attacker and the objective is to maximize the missile acceleration till the missile power bleed. 
We will  choose the escaping trajectory as a polynomial with unknown coefficient, then trying to find that coefficient which makes the missile exert a maximum acceleration to bleed its power as fast as can before it reach the target.  
% ===================================================
\section{Assumptions, Notation, and Nomenclature}
% ===================================================
\subsection*{Assumptions}

\begin{enumerate}
	\item Both the missile and target travel at constant velocity.
	\item Gravitational and drag effects have been neglected for simplicity.
\end{enumerate}
% ===================================================

% ===================================================
\subsection*{Notation}
% ===================================================
\begin{itemize}
	\item $n_c$ : Acceleration command (for the missile).
	\item $N'$ : Effective navigation ratio, a unit-less designer-chosen gain (usually in the range of $3\to5$).
	\item $V_c$ : missile-target closing velocity.
	\item $\lambda$ : line-of-sight angle.
	\item $\dot{\lambda}$ : line of sight rate.
	\item $R_{TM}$ : length of the line of sight.
	\item $L$ : missile lead angle.
	\item $HE$ : Heading error.
	\item $\dot{\beta}$ : angular velocity of the target.
	\item $V_{T1},V_{T2}$ : Target velocity components in the Earth fixed coordinate system.
	\item $V_{M1},V_{M2}$ : Missile velocity components in the Earth fixed coordinate system.
\end{itemize}
% ===================================================
\subsection*{Nomenclature}

\textbf{Inertial coordinate system:} fixed to the surface of a flat-Earth model ( the 1 axis is downrange and the 2 axis can either be altitude or cross-range).

\textbf{Missile lead angle:} theoretically correct angle
for the missile to be on a collision triangle with the target.

\textbf{Heading error ($HE$) :} angle represents the initial deviation of the missile from the collision triangle.

\textbf{line of sight:} The imaginary line connecting the missile and target.

\textbf{length of the line of sight ($R_{TM}$):} Instantaneous separation between missile and target.

\textbf{Miss distance :} The point of closest approach of the missile and target.

\textbf{Closing velocity ($V_c$):} the negative rate of change of the distance
from the missile to the target $Vc= -R_{TM} $.


% ===================================================


% ===================================================
\section{Proportional Navigation}
% ===================================================
In this section we will illustrate some fundamentals of missile guidance, focusing on the proportional navigation technique, which is one of the simplest guidance laws.

\subsection*{What is proportional navigation?}
the proportional navigation guidance law issues acceleration commands,
perpendicular to the instantaneous missile-target line-of-sight, which are
proportional to the line-of-sight rate and closing velocity. Mathematically, the
guidance law can be stated as

\begin{equation}
	n_c= N' V_c \dot{\lambda}
\end{equation}

where $n_c$ is the acceleration command (for the missile), $N'$ is the the effective navigation ratio, a unit-less designer-chosen gain (usually in the range of $3 \to 5$), $V_c$ is the missile-target closing velocity and $\lambda$ is the rate of the line-of-sight angle.


\subsection{Simulation of proportional navigation equations in 2-D}
In this topic we will introduce the equations of proportional navigation and the sequence to get a simulation for the path of the target and the attacker and how the missile acceleration will affect during this simulation.

\textbf{The simulation inputs} are the initial location of the missile ($R_{M1}, R_{M2}$) and the initial location of the target ($R_{T1}, R_{T2}$), target speed $V_T$, missile speed $V_M$, and effective navigation ratio $N'$.

There are two types of \textbf{error source} that cause the attacker miss the target; they are heading error ($HE$) and target maneuver ($n_T$), the acceleration of the target.

\subsubsection*{proportional navigation differential equations}

\textbf{component of target velocity} 
\begin{equation}
	V_{T1} = - V_T \cos(\beta)
\end{equation}

\begin{equation}
V_{T2} = - V_T \sin(\beta)
\end{equation}

\textbf{Relative missile-target separation}
\begin{equation}
	R_{TM1} = R{T1} - R_{M1}
\end{equation}
\begin{equation}
R_{TM2} = R{T2} - R_{M2}
\end{equation}

from the previous 2 equations we get
\begin{equation}
	R_{TM} = \sqrt{R_{TM1}^2 + R_{TM2}^2}
	\label{RTM}
\end{equation}

\textbf{line of sight angle}
\begin{equation}
	\lambda = \tan^{-1} (\dfrac{R_{TM2}}{R_{TM1}})
	\label{lambda}
\end{equation}

\textbf{missile lead angle} 
\begin{equation}
	L= \sin^{-1}(\dfrac{V_T \sin(\beta + \lambda)}{V_M})
\end{equation}

the angle between the downrange axis and $V_M$ vector is $\theta = \lambda + L$

\textbf{Missile velocity components} 

\begin{equation}
	V_{M1} = V_M \cos (\theta + HE)
\end{equation}

\begin{equation}
V_{M2} = V_M \sin (\theta + HE)
\end{equation}

\textbf{Relative velocity components}
\begin{equation}
	V_{TM1} = V_{T1} - V_{M1}
\end{equation}

\begin{equation}
V_{TM2} = V_{T2} - V_{M12}
\end{equation}


\textbf{closing velocity} it is the negative rate of change of the distance
from the missile to the target $Vc= -R_{TM} $, so we have to differentiate eq(\ref{RTM})

\begin{center}
	$\dot{R_{TM}}= \frac{1}{2} (R_{TM1}^2 + R_{TM2}^2)^{\frac{-1}{2}} [2 R_{TM1} \dot{R_{TM1}} + 2 R_{TM2} \dot{R_{TM2}}]$
\end{center}

we see that
%\begin{equation*}
% \dot{R_{TM1}=V_{TM1} , \dot{R_{TM2}=V_{TM2}
%\end{equation*}
%and 
%\begin{equation*}
%(R_{TM1}^2 + R_{TM2}^2)^{\frac{-1}{2}} = \dfrac{1}{R_{TM}}
%\end{equation*}
so we get 
\begin{equation}
	V_c = - \dot{R_{TM}} = - \dfrac{R_{TM1} V_{TM1}+R_{TM2} V_{TM2}}{R_{TM}}
\end{equation}

\textbf{line of sight rate} we have to differentiate eq(\ref{lambda}) using the rule $\tan^{-1}x = \frac{dx}{1+x^2}$ 

\begin{equation}
	\begin{split}
	\dot{\lambda} &= [\dfrac{1}{1+(\frac{R_{TM2}}{R_{TM1}})^2}] \dot{(\frac{R_{TM2}}{R_{TM1}})}\\
	&= \dfrac{R_{TM1}^2}{R_{TM1}^2 + R_{TM2}^2}[\dfrac{R_{TM1}\dot{R_{TM2}}- R_{TM2} \dot{R_{TM1}}}{R_{TM1}^2}]\\
	&=\dfrac{R_{TM1} V_{TM2} - R_{TM2} V_{TM1}}{R_{TM1}^2}
	\end{split}
\end{equation}

\textbf{magnitude of the missile guidance command}
\begin{equation}
		n_c= N' V_c \dot{\lambda}
\end{equation}

\textbf{missile acceleration components}
\begin{equation}
	a_{M1} = - n_c \sin \lambda
\end{equation}

\begin{equation}
a_{M2} = - n_c \cos \lambda
\end{equation}

\textbf{angular velocity of the target}
\begin{equation}
	\dot{\beta} = \dfrac{n_T}{V_T}
\end{equation}

we will solve all the equations in this section using second-order Runge–Kutta numerical integration procedure. If we have a first order differential equation of the form 
\begin{equation*}
	\dot{x} = f(x,t) 
\end{equation*} 
 where t is time, we seek to find a recursive relationship for x as a function of time.
 With the second-order Runge–Kutta numerical technique, the value of x at the
 next integration interval h is given by
 \begin{equation*}
 	x_{k+1} = x_k + \dfrac{hf(x,t)}{2} + \dfrac{h f(x, t+h)}{2}
 \end{equation*}
% ===================================================
\section{Models \& Simulations}
% ===================================================

In this section we will simulate equations in sec. 2.3 for proportional navigation using MATLAB and Simulink.
The following figure is a flowchart illustrating the steps of calculations using the equations in sec. 2.3  till we plot the trajectories.

\begin{figure}[htb]
	\centering
	\includegraphics[scale = 0.85]{fig/FlowchartPN.pdf}
	\caption{Flowchart illustrating the steps of calculations using the equations in sec. 2.3  till we plot the trajectories.}
	\label{flowchart PN}
\end{figure}


\subsection{MATLAB}
In this subsection we will use MATLAB to simulate equations in sec. 2.3 for proportional navigation. We will solve these differential equations using second-order Runge Kutta numerical integration technique, then we will draw the trajectories of the pursuit and evader for 4 cases for the target maneuver error source, and we will deduce the effect of the effective navigation ratio $N'$ and the other type of error source; Heading error. 
\subsubsection{Zero Target maneuver}
In this case the evader (Target - plane) don't do any effort to scape, it is just move in straight line, as we see in Fig. \ref{trajectory0N4}. So the pursuit (Attacker - Missile) don't have to bleed much energy to reach the target.

In the case of \textbf{zero Heading error} the effective navigation ratio has no effect on the simulation engagement at all. The missile's acceleration will be zero as in Fig. \ref{missile acceleration0N4}.

\begin{figure}[htb]
	\centering
	\includegraphics[scale = 0.75]{fig/trajectory0N4.pdf}
	\caption{Trajectory of the target and attacker in case of zero heading error and $N'=4$.}
	\label{trajectory0N4}
\end{figure}


\begin{figure}[htb]
	\centering
	\includegraphics[scale = 0.75]{fig/MissileAcceleration0N4.pdf}
	\caption{Missile acceleration in case of zero heading error and $N'=4$ .}
	\label{missile acceleration0N4}
\end{figure}


In the case of \textbf{Heading error = -20} increasing the effective navigation ratio causing heading error to be removed rabidly as we see from  Fig. \ref{trajectory20N3}, Fig. \ref{trajectory20N4} and Fig. \ref{trajectory20N5}. The effective navigation ratio has effect on the acceleration of the missile; the way that the missile will bleed energy as we see from Fig. \ref{missile acceleration20N3} , Fig. \ref{missile acceleration20N4} and Fig. \ref{missile acceleration20N5} the total acceleration (area under the curve) is increasing inversely proportional with the effective navigation ratio $N'$ .


\begin{figure}[htb]
	\centering
	\includegraphics[scale = 0.75]{fig/trajectory20N3.pdf}
	\caption{Trajectory of the target and attacker in case of heading error=-20 and $N'=3$.}
	\label{trajectory20N3}
\end{figure}

\begin{figure}[htb]
	\centering
	\includegraphics[scale = 0.75]{fig/MissileAcceleration20N3.pdf}
	\caption{Missile acceleration in case of heading error=-20 and $N'=3$ .}
	\label{missile acceleration20N3}
\end{figure}


\begin{figure}[htb]
	\centering
	\includegraphics[scale = 0.75]{fig/trajectory20N4.pdf}
	\caption{Trajectory of the target and attacker in case of heading error=-20 and $N'=4$.}
	\label{trajectory20N4}
\end{figure}


\begin{figure}[htb]
	\centering
	\includegraphics[scale = 0.75]{fig/MissileAcceleration20N4.pdf}
	\caption{Missile acceleration in case of heading error=-20 and $N'=4$ .}
	\label{missile acceleration20N4}
\end{figure}


\begin{figure}[htb]
	\centering
	\includegraphics[scale = 0.75]{fig/trajectory20N5.pdf}
	\caption{Trajectory of the target and attacker in case of heading error=-20 and $N'=5$.}
	\label{trajectory20N5}
\end{figure}


\begin{figure}[htb]
	\centering
	\includegraphics[scale = 0.75]{fig/MissileAcceleration20N5.pdf}
	\caption{Missile acceleration in case of heading error=-20 and $N'=5$ .}
	\label{missile acceleration20N5}
\end{figure}

%-----------------------------------------------------------

\subsubsection{Constant Target maneuver}
In this case the target do some effort to escape from the attacker, in form of constant acceleration (in our example target acceleration = $3g=96.6$).
We see from Fig. \ref{missile acceleration0NN3} and Fig. \ref{missile acceleration0NN5} that higher effective navigation ratio yields less acceleration to hit maneuvering target, and causes the missile to lead the target slightly more than lower effective navigation, as we see from Fig. \ref{trajectory0NN3} and Fig. \ref{trajectory0NN3}.



\begin{figure}[htb]
	\centering
	\includegraphics[scale = 0.75]{fig/trajectory0NN3.pdf}
	\caption{Trajectory of the target and attacker in case of heading error=0 and $N'=3$.}
	\label{trajectory0NN3}
\end{figure}


\begin{figure}[htb]
	\centering
	\includegraphics[scale = 0.75]{fig/MissileAcceleration0NN3.pdf}
	\caption{Missile acceleration in case of heading error=0 and $N'=3$ .}
	\label{missile acceleration0NN3}
\end{figure}



\begin{figure}[htb]
	\centering
	\includegraphics[scale = 0.75]{fig/trajectory0NN5.pdf}
	\caption{Trajectory of the target and attacker in case of heading error=0 and $N'=5$.}
	\label{trajectory0NN5}
\end{figure}


\begin{figure}[htb]
	\centering
	\includegraphics[scale = 0.75]{fig/MissileAcceleration0NN5.pdf}
	\caption{Missile acceleration in case of heading error=0 and $N'=5$ .}
	\label{missile acceleration0NN5}
\end{figure}
%-----------------------------------------------------------

\subsubsection{Polynomial Target maneuver}

%------------------------------------------------------------

\subsubsection{Trapezoidal Target maneuver}

%-----------------------------------------------------

\subsection{Simulink Models}

Add figures.

% ===================================================
\section{Genetic Algorithms Solution}
% ===================================================

% ===================================================
\section{Neural Networks Solution}
% ===================================================