% ===================================================
\section{Problem Statement}
% ===================================================
Evasion techniques.

Why use polynomials?

% ===================================================
\section{Assumptions, Notation, and Nomenclature}
% ===================================================
\subsection*{Assumptions}

\begin{enumerate}
	\item Both the missile and target travel at constant velocity.
	\item Gravitational and drag effects have been neglected for simplicity.
\end{enumerate}
% ===================================================

% ===================================================
\subsection*{Notation}
% ===================================================
\begin{itemize}
	\item $n_c$ : Acceleration command (for the missile).
	\item $N'$ : Effective navigation ratio, a unit-less designer-chosen gain (usually in the range of $3\to5$).
	\item $V_c$ : missile-target closing velocity.
	\item $\lambda$ : line-of-sight angle.
	\item $\dot{\lambda}$ : line of sight rate.
	\item $R_{TM}$ : length of the line of sight.
	\item $L$ : missile lead angle.
	\item $HE$ : Heading error.
	\item $\dot{\beta}$ : angular velocity of the target.
	\item $V_{T1},V_{T2}$ : Target velocity components in the Earth fixed coordinate system.
	\item $V_{M1},V_{M2}$ : Missile velocity components in the Earth fixed coordinate system.
\end{itemize}
% ===================================================
\subsection*{Nomenclature}

\textbf{Inertial coordinate system:} fixed to the surface of a flat-Earth model ( the 1 axis is downrange and the 2 axis can either be altitude or cross-range).

\textbf{Missile lead angle:} theoretically correct angle
for the missile to be on a collision triangle with the target.

\textbf{Heading error ($HE$) :} angle represents the initial deviation of the missile from the collision triangle.

\textbf{line of sight:} The imaginary line connecting the missile and target.

\textbf{length of the line of sight ($R_{TM}$):} Instantaneous separation between missile and target.

\textbf{Miss distance :} The point of closest approach of the missile and target.

\textbf{Closing velocity ($V_c$):} the negative rate of change of the distance
from the missile to the target $Vc= -R_{TM} $.


% ===================================================


% ===================================================
\section{Proportional Navigation Equations}
% ===================================================
In this section we will illustrate some fundamentals of missile guidance, focusing on the proportional navigation technique, which is one of the simplest guidance laws.

\subsection*{What is proportional navigation?}
the proportional navigation guidance law issues acceleration commands,
perpendicular to the instantaneous missile-target line-of-sight, which are
proportional to the line-of-sight rate and closing velocity. Mathematically, the
guidance law can be stated as

\begin{equation}
	n_c= N' V_c \dot{\lambda}
\end{equation}

where $n_c$ is the acceleration command (for the missile), $N'$ is the the effective navigation ratio, a unit-less designer-chosen gain (usually in the range of $3 \to 5$), $V_c$ is the missile-target closing velocity and $\lambda$ is the rate of the line-of-sight angle.


\subsection*{Simulation of proportional navigation in 2-D}

% ===================================================
\section{Models \& Simulations}
% ===================================================

\subsection{MATLAB}
MATLAB Code: trapezoidal versus polynomials.

plots.
\subsection{Simulink Models}

Add figures.

% ===================================================
\section{Genetic Algorithms Solution}
% ===================================================

% ===================================================
\section{Neural Networks Solution}
% ===================================================