% ===================================================
\section{Problem Statement}
% ===================================================
Evasion techniques.

Why use polynomials?

% ===================================================
\section{Assumptions, Notation, and Nomenclature}
% ===================================================
\subsection*{Assumptions}

\begin{enumerate}
	\item Both the missile and target travel at constant velocity.
	\item Gravitational and drag effects have been neglected for simplicity.
\end{enumerate}
% ===================================================

% ===================================================
\subsection*{Notation}
% ===================================================
\begin{itemize}
	\item $n_c$ : Acceleration command (for the missile).
	\item $N'$ : Effective navigation ratio, a unit-less designer-chosen gain (usually in the range of $3\to5$).
	\item $V_c$ : missile-target closing velocity.
	\item $\lambda$ : line-of-sight angle.
	\item $\dot{\lambda}$ : line of sight rate.
	\item $R_{TM}$ : length of the line of sight.
	\item $L$ : missile lead angle.
	\item $HE$ : Heading error.
	\item $\dot{\beta}$ : angular velocity of the target.
	\item $V_{T1},V_{T2}$ : Target velocity components in the Earth fixed coordinate system.
	\item $V_{M1},V_{M2}$ : Missile velocity components in the Earth fixed coordinate system.
\end{itemize}
% ===================================================
\subsection*{Nomenclature}

\textbf{Inertial coordinate system:} fixed to the surface of a flat-Earth model ( the 1 axis is downrange and the 2 axis can either be altitude or cross-range).

\textbf{Missile lead angle:} theoretically correct angle
for the missile to be on a collision triangle with the target.

\textbf{Heading error ($HE$) :} angle represents the initial deviation of the missile from the collision triangle.

\textbf{line of sight:} The imaginary line connecting the missile and target.

\textbf{length of the line of sight ($R_{TM}$):} Instantaneous separation between missile and target.

\textbf{Miss distance :} The point of closest approach of the missile and target.

\textbf{Closing velocity ($V_c$):} the negative rate of change of the distance
from the missile to the target $Vc= -R_{TM} $.


% ===================================================


% ===================================================
\section{Proportional Navigation}
% ===================================================
In this section we will illustrate some fundamentals of missile guidance, focusing on the proportional navigation technique, which is one of the simplest guidance laws.

\subsection*{What is proportional navigation?}
the proportional navigation guidance law issues acceleration commands,
perpendicular to the instantaneous missile-target line-of-sight, which are
proportional to the line-of-sight rate and closing velocity. Mathematically, the
guidance law can be stated as

\begin{equation}
	n_c= N' V_c \dot{\lambda}
\end{equation}

where $n_c$ is the acceleration command (for the missile), $N'$ is the the effective navigation ratio, a unit-less designer-chosen gain (usually in the range of $3 \to 5$), $V_c$ is the missile-target closing velocity and $\lambda$ is the rate of the line-of-sight angle.


\subsection{Simulation of proportional navigation equations in 2-D}
In this topic we will introduce the equations of proportional navigation and the sequence to get a simulation for the path of the target and the attacker and how the missile acceleration will affect during this simulation.

\textbf{The simulation inputs} are the initial location of the missile ($R_{M1}, R_{M2}$) and the initial location of the target ($R_{T1}, R_{T2}$), target speed $V_T$, missile speed $V_M$, and effective navigation ratio $N'$.

There are two types of \textbf{error source} that cause the attacker miss the target; they are heading error ($HE$) and target maneuver ($n_T$), the acceleration of the target.

\subsubsection*{proportional navigation differential equations}

\textbf{component of target velocity} 
\begin{equation}
	V_{T1} = - V_T \cos(\beta)
\end{equation}

\begin{equation}
V_{T2} = - V_T \sin(\beta)
\end{equation}

\textbf{Relative missile-target separation}
\begin{equation}
	R_{TM1} = R{T1} - R_{M1}
\end{equation}
\begin{equation}
R_{TM2} = R{T2} - R_{M2}
\end{equation}

from the previous 2 equations we get
\begin{equation}
	R_{TM} = \sqrt{R_{TM1}^2 + R_{TM2}^2}
	\label{RTM}
\end{equation}

\textbf{line of sight angle}
\begin{equation}
	\lambda = \tan^{-1} (\dfrac{R_{TM2}}{R_{TM1}})
	\label{lambda}
\end{equation}

\textbf{missile lead angle} 
\begin{equation}
	L= \sin^{-1}(\dfrac{V_T \sin(\beta + \lambda)}{V_M})
\end{equation}

the angle between the downrange axis and $V_M$ vector is $\theta = \lambda + L$

\textbf{Missile velocity components} 

\begin{equation}
	V_{M1} = V_M \cos (\theta + HE)
\end{equation}

\begin{equation}
V_{M2} = V_M \sin (\theta + HE)
\end{equation}

\textbf{Relative velocity components}
\begin{equation}
	V_{TM1} = V_{T1} - V_{M1}
\end{equation}

\begin{equation}
V_{TM2} = V_{T2} - V_{M12}
\end{equation}


\textbf{closing velocity} it is the negative rate of change of the distance
from the missile to the target $Vc= -R_{TM} $, so we have to differentiate eq(\ref{RTM})

\begin{center}
	$\dot{R_{TM}}= \frac{1}{2} (R_{TM1}^2 + R_{TM2}^2)^{\frac{-1}{2}} [2 R_{TM1} \dot{R_{TM1}} + 2 R_{TM2} \dot{R_{TM2}}]$
\end{center}

we see that
%\begin{equation*}
% \dot{R_{TM1}=V_{TM1} , \dot{R_{TM2}=V_{TM2}
%\end{equation*}
%and 
%\begin{equation*}
%(R_{TM1}^2 + R_{TM2}^2)^{\frac{-1}{2}} = \dfrac{1}{R_{TM}}
%\end{equation*}
so we get 
\begin{equation}
	V_c = - \dot{R_{TM}} = - \dfrac{R_{TM1} V_{TM1}+R_{TM2} V_{TM2}}{R_{TM}}
\end{equation}

\textbf{line of sight rate} we have to differentiate eq(\ref{lambda}) using the rule $\tan^{-1}x = \frac{dx}{1+x^2}$ 

\begin{equation}
	\begin{split}
	\dot{\lambda} &= [\dfrac{1}{1+(\frac{R_{TM2}}{R_{TM1}})^2}] \dot{(\frac{R_{TM2}}{R_{TM1}})}\\
	&= \dfrac{R_{TM1}^2}{R_{TM1}^2 + R_{TM2}^2}[\dfrac{R_{TM1}\dot{R_{TM2}}- R_{TM2} \dot{R_{TM1}}}{R_{TM1}^2}]\\
	&=\dfrac{R_{TM1} V_{TM2} - R_{TM2} V_{TM1}}{R_{TM1}^2}
	\end{split}
\end{equation}

\textbf{magnitude of the missile guidance command}
\begin{equation}
		n_c= N' V_c \dot{\lambda}
\end{equation}

\textbf{missile acceleration components}
\begin{equation}
	a_{M1} = - n_c \sin \lambda
\end{equation}

\begin{equation}
a_{M2} = - n_c \cos \lambda
\end{equation}

\textbf{angular velocity of the target}
\begin{equation}
	\dot{\beta} = \dfrac{n_T}{V_T}
\end{equation}

we will solve all the equations in this section using second-order Runge–Kutta numerical integration procedure. If we have a first order differential equation of the form 
\begin{equation*}
	\dot{x} = f(x,t) 
\end{equation*} 
 where t is time, we seek to find a recursive relationship for x as a function of time.
 With the second-order Runge–Kutta numerical technique, the value of x at the
 next integration interval h is given by
 \begin{equation*}
 	x_{k+1} = x_k + \dfrac{hf(x,t)}{2} + \dfrac{h f(x, t+h)}{2}
 \end{equation*}
% ===================================================
\section{Models \& Simulations}
% ===================================================

\subsection{MATLAB}
MATLAB Code: trapezoidal versus polynomials.

plots.
\subsection{Simulink Models}

Add figures.

% ===================================================
\section{Genetic Algorithms Solution}
% ===================================================

% ===================================================
\section{Neural Networks Solution}
% ===================================================