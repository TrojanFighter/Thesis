A circle is the locus moving at a constant distance (called the circle's radius $r$) from a fixed point (called the circle's centre $O$).
In the limit of an infinite radius $(r\longrightarrow\infty∞)$, the circle degenerates into a straight line.
Another definition of the circle is that it is the locus of a point moving such that the ratio of its distances from two fixed points A and B is a constant $k$. With this definition, the circle is called a circle of Apollonius (in honour of Apollonius of Perga (ca. 262-190 BC), the Great Geometer of Antiquity). In the limit ($k\longrightarrow1$) this circle degenerates into the perpendicular bisector of $\overline{AD}$, while in the two limits ($k\longrightarrow0$) and ($k\longrightarrow\infty$), this circle collapses to the two points A and B, respectively. \\

Two Apollonius circles are studied herein, namely the $AD$ Apollonius circle, and the $TA$ Apollonius circle. The study will be based on simple and intuitive plane-geometric arguments and will avoid the more involved treatment of analytic geometry. Figures \ref{1} and \ref{2} shows the Apollonius circles for a moving point $P$ such that $\dfrac{AP}{PB}=k\neq1$. The case $k>1$ is shown in Fig. \ref{1}, while the case $k<1$ is shwon in Fig. \ref{2}. The points $I$ and $E$ are the two special cases of $P$ that lie on the straight line extension of the straight segment $\overline{AB}$. These points divide the straight segment $\overline{AB}$ internally and externally in the ratio $k\ (k\neq1)$, i.e., 

\begin{equation}
  %\tag{*}
  \boxed{
  \dfrac{AI}{IB}=\dfrac{AE}{EB}=k,\ \{k\neq1\}.}
  \label{eqn:kratio}
\end{equation} 


\begin{figure}[htb]
\centering
\includegraphics[scale = 0.5]{fig/drawing1.pdf}
\caption{Apollonius circle for a moving point $P$ such that $\dfrac{AP}{PB}=k>1$. Here $m\angle API = m\angle BPI$ and $m\angle A'PE = m\angle BPE$.}
\label{1}
\end{figure}

\begin{figure}[htb]
\centering
\includegraphics[scale = 0.5]{fig/drawing2.pdf}
\caption{Apollonius circle for a moving point $P$ such that $\dfrac{AP}{PB}=k<1$. Here $m\angle API = m\angle BPI$ and $m\angle APE = m\angle B'PE$.}
\label{2}
\end{figure}

If we denote the position vector of a point by a bold version of its name, we can rewrite \eqref{eqn:kratio} in vectorial form as

\begin{equation}
(\boldsymbol{I}-\boldsymbol{A})=k(\boldsymbol{B}-\boldsymbol{I}),
\end{equation}

\begin{equation}
(\boldsymbol {E}-\boldsymbol{A})=k(\boldsymbol{E}-\boldsymbol{B}),
\end{equation}

which can be used to express $\boldsymbol{I}$ and $\boldsymbol{E}$ in terms of $\boldsymbol{A}$ and $\boldsymbol{B}$ as 
\begin{equation}
\boldsymbol{I} = \dfrac{1}{k+1} (\boldsymbol{A}+k\boldsymbol{B}),
\label{eqn:ipoint}
\end{equation}

\begin{equation}
\boldsymbol{E} = \dfrac{1}{k-1} (-\boldsymbol{A}+k\boldsymbol{B}).
\label{eqn:epoint}
\end{equation}

The Apollonius circle is easily characterized by the points $\boldsymbol{I}$ and $\boldsymbol{E}$, since they are the two end points of one of the its diameters. The center of the circle is the midpoint of points $\boldsymbol{I}$ and $\boldsymbol{E}$, namely:

\begin{equation}
\begin{split}
\boldsymbol{O} & = \dfrac{1}{2} (\boldsymbol{I}+\boldsymbol{E})\\
& = \dfrac{1}{2} [(\dfrac{1}{k+1}-\dfrac{1}{k-1})]\boldsymbol{A}+k(\dfrac{1}{k+1}+\dfrac{1}{k-1}) \boldsymbol{B}\\
& =-\dfrac{1}{k^{2}-1}\boldsymbol{A} + \dfrac{{k^{2}}}{k^{2}-1} \boldsymbol{B},
\end{split}
\label{eqn:center}
\end{equation}

while the radius of the circle is half the length of the displacement from $\boldsymbol{I}$ to $\boldsymbol{E}$,
\begin{equation}
\begin{split}
r & =\dfrac{1}{2} \lvert \boldsymbol{I} -\boldsymbol{E}\rvert \\
& = \dfrac{1}{2} \lvert (\dfrac{1}{k+1}+\dfrac{1}{k-1})\boldsymbol{A}+k(\dfrac{1}{k+1}-\dfrac{1}{k-1}) \boldsymbol{B}\rvert \\
& =  \lvert\dfrac{k}{k^{2}-1}\boldsymbol{A} - \dfrac{k}{k^{2}-1} \boldsymbol{B}\rvert\\
& = \lvert\dfrac{k}{k^{2}-1}\rvert \lvert\boldsymbol{A} -\boldsymbol{B}\rvert = \dfrac{k}{k^{2}-1}(AB).
\end{split}
\label{eqn:radius}
\end{equation}

It is clear from \eqref{eqn:center} and \eqref{eqn:radius} that $\lim_{k\to1}\lvert\boldsymbol{O}\rvert\to\infty$ and $\lim_{k\to1}r\to\infty$, and hence for $k=1$, the Apollonius circle degenerates into a straight line, namely the perpendicular bisector of the straight segment $\overline{AB}$ (Fig. \ref{3}).

\begin{figure}[htb]
\centering
\includegraphics[scale = 0.5]{fig/drawing3.pdf}
\caption{For $k=1$, the Apollonius circle in figures \ref{1} or \ref{2} degenerates into the perpendicular bisector of the straight segment $\overline{AB}$. The point $E$ disappears in this figure as it goes to $\infty$  }
\label{3}
\end{figure}


\section{The $AD$ Apollonius circle}
For the $AD$ Apollonius circle, the two fixed points are the initial positions of the Attacker $\boldsymbol{A}=(x_{A},0)$ and the initial position of the defender $\boldsymbol{D}=(-x_{A},0)$. The fixed ratio of the circle $k$ is replaced by the following dimensionless ratio which is the Attacker's speed normalized w.r.t the Defender speed: 
\begin{equation}
\gamma = \dfrac{V_{A}}{V_{D}}.
\end{equation}
We will consider the three cases of 

\begin{enumerate}
\item $\gamma<1$ (fast Defender) discussed in Garcia et al. \cite{garcia2015active}.
\item $\gamma =1$ (same-speed Defender) discussed in Garcia et al. \cite{pachter2014active, garcia2015escape}.
\item $\gamma>1$ (slow Defender), which is a novel case. 
\end{enumerate}

Substituting the values of $\boldsymbol{A}$ and $\boldsymbol{D}$ above for $\boldsymbol{A}$ and $\boldsymbol{B}$ in \eqref{eqn:ipoint},\eqref{eqn:epoint},\eqref{eqn:center} and \eqref{eqn:radius}, respectively, and replacing $k$ therein by $\gamma$ we obtain for $\gamma\neq1$,

\begin{eqnarray}
\boldsymbol{I_{1}} &=& (\dfrac{1-\gamma}{1+\gamma}x_{A},0),\\
\boldsymbol{E_{1}} &=& (\dfrac{1+\gamma}{1-\gamma}x_{A},0),\\
\boldsymbol{O_{1}} &=& (\dfrac{1+\gamma^{2}}{1-\gamma^{2}}x_{A},0),\\
\label{O1}
r_{1} &=& \dfrac{2\gamma}{\lvert1-\gamma^{2}\rvert}x_{A}.
\label{r1}
\end{eqnarray}

Three interesting sets of limits are now considered. The first set of limits are those when ($\gamma\to1$), namely 

\begin{eqnarray}
\lim_{\gamma\to1} \boldsymbol{I}_1 &=& (0,0),\\
\lim_{\gamma\to1} \boldsymbol{E}_1 &=& (\mp\infty,0),\\
\lim_{\gamma\to1} \boldsymbol{O}_1 &=& (\mp\infty,0),\\
\lim_{\gamma\to1} r_1 &=& \infty,
\end{eqnarray}

which means that the $AD$ Apollonius circle degenerates in the limit ($\gamma\to1$) to a circle of an infinite radius with center still on the $x-$axis but infinitely distant from the origin. More precisely, this limit is identified as a straight line, namely the perpendicular bisector of the straight-line segment $\overline{AD}$.

The second set of limits are those when ($\gamma\to0$), namely 

\begin{eqnarray}
\lim_{\gamma\to0} \boldsymbol{I}_1 &=& (x_A,0),\\
\lim_{\gamma\to0} \boldsymbol{E}_1 &=& (x_A,0),\\
\lim_{\gamma\to0} \boldsymbol{O}_1 &=& (x_A,0),\\
\lim_{\gamma\to0} r_1 &=& 0,
\end{eqnarray}

which means that the $AD$ Apollonius circle collapses in the limit ($\gamma\to0$) to a single point, namely the initial position of the Attacker $\boldsymbol{A}=(x_A,0)$.

The third set of limits are those when ($\gamma\to\infty$), namely
\begin{eqnarray}
\lim_{\gamma\to\infty} \boldsymbol{I}_1 &=& (-x_A,0), \\
\lim_{\gamma\to\infty} \boldsymbol{E}_1 &=& (-x_A,0),\\
\lim_{\gamma\to\infty} \boldsymbol{O}_1 &=& (-x_A,0),\\
\lim_{\gamma\to\infty} r_1 &=& 0,
\end{eqnarray}

which means that the $AD$ Apollonius circle collapses in the limit ($\gamma\to\infty$) to a single point, namely the initial position of the Defender $\boldsymbol{D}=(-x_A,0)$.

\section{The $TA$ Apollonius circle}
For the $TA$ Apollonius circle, the two fixed points are the initial position of the Target $\boldsymbol{T}=(x_{T},y_{T})$ and the initial position of the Attacker  $\boldsymbol{A}=(x_{A},0)$.
Again, the fixed ratio of the circle $k$ is replaced by a dimensionless quantity, namely the Target's speed normalized w.r.t. the Attacker's speed: 
\begin{equation}
\alpha= \dfrac{V_{T}}{V_{A}}.
\end{equation}

Here, we will consider only the case $\alpha <1$, since for $\alpha\geqslant1$, the Target always survives, even without any assistance from the Defender.

Now, we substitute the values of $\boldsymbol{T}$ and $\boldsymbol{A}$ above for $\boldsymbol{A}$ and $\boldsymbol{B}$ in \eqref{eqn:ipoint},\eqref{eqn:epoint},\eqref{eqn:center} and \eqref{eqn:radius}, respectively, and replace $k$ therein by $\alpha$ to obtain:

\begin{equation}
\boldsymbol{I_{2}} =\dfrac{1}{1+\alpha}(\boldsymbol{T}+\alpha \boldsymbol{A}) =\dfrac{1}{1+\alpha}(x_{T}+\alpha x_{A},y_{T}),
\end{equation}

\begin{equation}
\boldsymbol{E_{2}} =\dfrac{1}{\alpha-1}(-\boldsymbol{T}+\alpha \boldsymbol{A}) =\dfrac{1}{1-\alpha}(x_{T}-\alpha x_{A},y_{T}),
\end{equation}

\begin{equation}
\boldsymbol{O_{2}} =\dfrac{1}{1-\alpha^{2}}(\boldsymbol{T}-\alpha^{2} \boldsymbol{A}) =\dfrac{1}{1-\alpha^{2}}(x_{T}-\alpha^{2} x_{A},y_{T}),
\label{O2}
\end{equation}

\begin{equation}
r_{2} =\dfrac{\alpha}{1-\alpha^{2}}(TA)
 = \dfrac{\alpha d}{1-\alpha^{2}},
 \label{r2}
\end{equation}

where $d=TA=$ the initial distance between the target and Attacker, namely
\begin{equation}
d=\sqrt{(x_{T}-x_{A})^{2}+{y_{T}}^{2}}.
\label{d}
\end{equation}
